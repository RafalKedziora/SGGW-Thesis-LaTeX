\documentclass{SGGW-thesis}

\INZYNIERSKAtrue
\WZIMtrue

\title{Stworzenie klasy \LaTeX-owej\\do użytku przy pisaniu\\pracy dyplomowej w~SGGW}
\Etitle{Creation of the \LaTeX\ Class to Use When Writing a Thesis\\at the Warsaw University of Life Sciences -- SGGW}
\author{Łukasz Adamczyk}
\date{2017\footnote{Dokument skompilowano z klasą {\tt SGGW-thesis} w wersji \version. Aktualną wersję klasy można pobrać ze strony \url{http://stud.lchmiel.pl} $\rightarrow$
Seminarium dyplomowe lub z \url{https://github.com/Shady6/SGGW-Thesis-Latex}. Proszę przeczytać plik {\tt readme}. W~Państwa pracach proszę usunąć tę notkę.}}
\album{154833}
\thesis{Praca dyplomowa na kierunku:}
\course{Informatyka}
\promotor{dr.\ hab.\ inż.\ Leszka Chmielewskiego, prof.\ SGGW}
\pworkplace{Wydział Zastosowań Informatyki i Matematyki\\Katedra Informatyki}

\usepackage{hyperref}

\begin{document}
\maketitle
\statementpage
\abstractpage
{Stworzenie klasy \LaTeX-owej do użytku przy pisaniu pracy dyplomowej w SGGW}
{Tematem niniejszej pracy było zaimplementowanie klasy \LaTeX-owej pozwalającej na formatowanie tekstu zgodnie z wytycznymi nałożonymi przez uczelnię. Praca zawiera dwie
główne części. Pierwsza z nich zawiera opis najważniejszych aspektów implementacji klasy. Natomiast druga część skupia się na sposobie użycia klasy przez osoby piszące prace
dyplomowe.}
{LaTeX, klasa, praca dyplomowa, implementacja, SGGW, Szkoła Główna Gospodarstwa Wiejskiego}
{Creation of the \LaTeX\ Class to be Used When Writing a Thesis at the Warsaw University of Life Sciences -- SGGW}
{The subject of this study was to implement a \LaTeX\ class that allows for text formatting according to the guidelines imposed by the University. The work consists of two
main parts. The first one describes the most important aspects of the implementation. The second part focuses on how to use the class by people writing the theses.}
{LaTeX, class, thesis, implementation, SGGW, Warsaw University of Life Sciences}


{
  % Spis treści może być złożony z pojedynczą interlinią, np. jeśli jedna linia wychodzi na następną stronę.
  % W przeciwnym razie spis treści wstawić bez powyższego rozkazu i klamry.
  \doublespacing
  \tableofcontents
}

\startchapterfromoddpage % niezależnie od długości spisu treści pierwszy rozdział zacznie się na nieparzystej stronie

\chapter{Wstęp}
Pisanie pracy dyplomowej oprócz oczywistego aspektu przekazywania informacji wiąże się także z odpowiednim sformatowaniem wynikowego tekstu według zasad narzuconych przez
uczelnię. Jednym z narzędzi ułatwiających formatowanie jest \LaTeX, system składania dokumentów opierający się na paradygmacie WYSIWYM ({\em ang.\ What You See Is What You
Mean}) co oznacza, że pisząc tekst określamy rolę dla każdego fragmentu danych, a~\LaTeX zajmuje się odpowiednią prezentacją. W przypadku Szkoły Głównej Gospodarstwa
Wiejskiego w~Warszawie oczekiwania pod względem formatowania dokumentu silnie odbiegają od standardowych rozwiązań zaimplementowanych w~\LaTeX-u, dlatego postanowiłem
stworzyć klasę {\tt SGGW-thesis}, która pomaga w~stworzeniu pracy dyplomowej spełniającej oczekiwania Uczelni.


\section{Cel i zakres pracy}
Praca ma na celu zaimplementowanie klasy \LaTeX-owej, której użycie znacznie ułatwia formatowanie pracy dyplomowej pisanej w~Szkole Głównej Gospodarstwa Wiejskiego. Wynikową
klasę cechować będzie zwięzłość oraz czytelność. Cechy te powinny ułatwić zmianę implementacji w przypadku modyfikacji wymagań stawianych przez uczelnię. Treść pracy zawierać
będzie opis najważniejszych fragmentów implementacji oraz przewodnik użycia klasy przy pisaniu prac dyplomowych. Ponadto ta praca inżynierska zostanie napisana przy użyciu
opisywanej klasy i stanowić będzie swoistą wizytówkę jej działania.


\chapter{Implementacja}
W tym rozdziale przedstawiono techniczne aspekty implementacji. Większość wiedzy na temat implementacji własnej klasy \LaTeX-owej zaczerpnięto z~\cite{latexclass}. Opisywana
klasa została oparta na standardowej klasie \LaTeX-owej \verb|report|, z której dziedziczy zachowanie nieokreślone w kodzie źródłowym. Główne elementy implementacji,
pozwalającej na odpowiednie formatowanie dokumentu to:
\begin{enumerate} [label=\alph*.]
\item{implementacja dodatkowych zmiennych,}
\item{reimplementacja komendy \verb|\maketitle|,}
\item{implementacja dodatkowych komend: \verb|\twoppage|, \verb|\threeppage|
\verb|\statementpage|, \verb|\abstractpage| oraz \verb|\beforelastpage|,}
\item{konfiguracja pakietów standardowych \LaTeX-a.}
\end{enumerate}
Powyższe elementy opisano w~rozdziałach~\ref{sec:dodatkowe-zmienne}-\ref{sec:konfiguracja-pakietow}.

Klasa zawiera teksty w języku polskim, z polskimi znakami. Wszystkie polskie znaki sa generowane odpowiednimi makrami \TeX-a, na przykład  ć=\verb|\'{c}|, ł=\verb|{\l}|,
ą=\verb|\k{a}|. Dzięki temu jest ona zapisana w pliku w kodzie ASCII, co zabezpiecza ją przed uszkodzeniem w przypadku przenoszenia między różnymi platformami. Oczywiście,
najwygodniej jest pisać samą prace dyplomową w kodowaniu {\tt UTF8} z polskimi znakami wprowadzanymi wprost z~klawiatury, co jest wpieranie przez klasę.


\section{Dodatkowe zmienne}
\label{sec:dodatkowe-zmienne}
Standardowy dokument \LaTeX-owy w swojej preambule pozwala na przypisanie zmiennych takich jak:
\begin{enumerate}[label=\alph*.]
\item{\verb|\title| -- tytuł dokumentu,}
\item{\verb|\author| -- autor dokumentu,}
\item{\verb|\date| -- data dokumentu,}
\end{enumerate}
które wykorzystywane są do tworzenia strony tytułowej. Opisywana klasa rozszerza tę funkcjonalność poprzez dodanie zmiennych:
\begin{enumerate}[label=\alph*.]
\item{\verb|\university| -- nazwa uczelni,}
\item{\verb|\dep| -- nazwa wydziału,}
\item{\verb|\Etitle| -- tytuł dokumentu w języku angielskim,}
\item{\verb|\album| -- numer albumu autora,}
\item{\verb|\thesis| -- rodzaj pracy dyplomowej,}
\item{\verb|\course| -- nazwa kierunku,}
\item{\verb|\promotor| -- dane promotora pracy,}
\item{\verb|\pworkplace| -- miejsce pracy promotora,}
\end{enumerate}
które wykorzystywane są w komendzie \verb|\maketitle| opisanej w podrozdziale~\ref{sec:maketitle}.


\section{Reimplementacja komendy \textbackslash maketitle}
\label{sec:maketitle}
Tytułowa klasa pozwala na użycie standardowej komendy \verb|\maketitle| do wygenerowania strony tytułowej pracy dyplomowej dzięki zastąpieniu konwencjonalnej implementacji.
W tym celu użyto komendy \verb|\renewcommand{\maketitle}|.
Większość tekstu używanego przez tę komendę pochodzi ze zmiennych opisanych w poprzednim podrozdziale. Wyśrodkowanie tekstu uzyskano za pomocą środowiska \verb|center|,
natomiast do wyrównania do prawej strony użyto środowiska \verb|flushright|. Rozstawienie pionowe jest wynikiem wielokrotnego użycia komendy \verb|\vspace| pozwalającej na
wstawienie pustej przestrzeni o wysokości zależnej od podanego parametru. Komenda \verb|\maketitle| generuje drugą stronę jako pustą, chyba że podczas deklaracji klasy
dokumentu użyto opcjonalnego parametru \verb|[multip]|. Takie zachowanie osiągnięto dzięki sprawdzeniu \verb|\if@multip|.


\section{Implementacja dodatkowych komend}
\label{sec:implementacja-dodatkowych}
Klasa {\tt SGGW-thesis} pozwala na generowanie standardowych stron wymaganych w~pracy dyplomowej przez SGGW poprzez implementację komend:
\begin{enumerate}[label=\alph*.,itemsep=0pt]
\item{\verb|\statementpage| -- trzecia strona dokumentu zawierająca oświadczenia,}
\item{\verb|\abstractpage| -- piąta strona dokumentu zawierająca streszczenia,}
\item{\verb|\twoppage| -- opcjonalna strona druga dla prac dwuosobowych,}
\item{\verb|\threeppage| -- opcjonalna strona druga dla prac trzyosobowych,}
\item{\verb|\beforelastpage| -- przedostatnia strona.}
\end{enumerate}
W powyższych komendach pionowe rozmieszczenie elementów zostało osiągnięte poprzez wykorzystanie \verb|\vfill|, które wypełnia miejsce użycia pustą przestrzenią równomiernie
dla każdego użycia na danej stronie. Wykropkowane miejsca na ręczny podpis są zrealizowane za pomocą \verb|\dotfill|. Dodatkowo \verb|\statementpage| wykorzystuje wspomniane
wcześniej sprawdzenie \verb|\if@multip| do wybrania odpowiedniej wersji tekstu.


\section{Konfiguracja pakietów \LaTeX-a}
\label{sec:konfiguracja-pakietow}
W klasach wykorzystywanie pakietów odbywa się za pomocą komendy \verb|RequirePackage|. Pakiety używane są do rozszerzenia lub zmiany funkcjonalności. Opisywana klasa korzysta
z następujących pakietów:
\begin{enumerate}[label=\alph*.,itemsep=0pt]
\item{[\verb|T1]{fontenc}| -- odpowiednie drukowanie polskich znaków,}
\item{\verb|{mathptmx}| -- czcionka Times New Roman,}
\item{\verb|{polski}| -- zbiór zasad pisania i dzielenia wyrazów w języku polskim,}
\item{\verb|[utf8]{inputenc}| -- odpowiednie wpisywanie polskich znaków,}
\item{\verb|[nottoc,numbib]{tocbibind}| -- umieszczenie bibliografii wraz z numerem w spisie treści oraz usunięcie ze spisu treści odniesienia do niego samego,}
\item{\verb|{xifthen}| -- umożliwienie wykorzystania komendy \verb|\ifthenelse|,}
\item{\verb|[a4paper,top=2.5cm,bottom=2.5cm,inner=3.5cm,outer=2cm]{geometry}|\\ -- ustawienie rozmiaru papieru oraz wielkości marginesów.}
\end{enumerate}


\chapter{Użycie}
Ten rozdział może być traktowany jako swoista instrukcja dla studentów SGGW, którzy chcą wykorzystać opisywaną klasę przy pisaniu pracy dyplomowej. Wymagana będzie wiedza o
tworzeniu dokumentów w \LaTeX-u, polecanym źródłem zdobycia takiej wiedzy jest~\cite{latexcompanion}. Pierwszy podrozdział opisuje jak zainstalować klasę w systemie
operacyjnym Windows, a dokładniej w dystrybucji MiK\TeX. Drugi podrozdział przedstawia  potrzebne deklaracje, wykorzystywane w dokumencie, natomiast trzeci podrozdział
demonstruje użycie komend wewnątrz środowiska \verb|document|.


\section{Instalacja}
Aby zainstalować klasę w MiK\TeX-u, należy stworzyć w systemie folder, w którym będą przechowywane lokalne pliki. Warto zaznaczyć, że folder ten nie może być podkatalogiem
głównego katalogu MiK\TeX-a. Kolejnym krokiem jest zarejestrowanie stworzonego przez nas folderu jako lokalnego źródła MiK\TeX-a. Aby to uczynić w menu \verb|Start|
znajdujemy oraz uruchamiamy \verb|MiKTeX Settings (Admin)|, przechodzimy do zakładki \verb|Roots|, klikamy przycisk \verb|Add...| i wybieramy stworzony folder, zatwierdzając
przyciskiem \verb|OK|. Następnie należy stworzyć drzewo podfolderów zgodne z~TDS (\TeX Directory Structure)~\cite{tds}. Przykładem może być \verb|\latex\cls|. Umieszczamy
plik \verb|SGGW-thesis.cls| w folderze \verb|cls| i wracamy do ustawień MiK\TeX-a. Tym razem przechodzimy do zakładki \verb|General| i naciskamy przycisk \verb|Refresh FNDB|.


\section{Preambuła}
Dokument \LaTeX-owy rozpoczynany jest preambułą, w której dokonujemy deklaracji m.in. klas, pakietów oraz zmiennych. Aby użyć klasy {\tt SGGW-thesis} należy umieścić komendę
\verb|\documentclass{SGGW-thesis}| na początku dokumentu. W~przypadku pracy pisanej przez kilka osób należy skorzystać z opcjonalnego parametru \verb|multip|, który
spowoduje, że strona druga nie zostanie automatycznie wygenerowana jako pusta.  Następnie należy przypisać wartości zmiennym używanym do stworzenia strony tytułowej:
\begin{enumerate}[label=\alph*.]
\item{\verb|\title| -- tytuł dokumentu,}
\item{\verb|\author| -- autor dokumentu,}
\item{\verb|\date|  -- data dokumentu,}
\item{\verb|\university| -- nazwa uczelni,}
\item{\verb|\dep| -- nazwa wydziału,}
\item{\verb|\Etitle| -- tytuł dokumentu w języku angielskim,}
\item{\verb|\album| -- numer albumu autora,}
\item{\verb|\thesis| -- rodzaj pracy dyplomowej,}
\item{\verb|\course| -- nazwa kierunku,}
\item{\verb|\promotor| -- dane promotora pracy,}
\item{\verb|\pworkplace| -- miejsce pracy promotora.}
\end{enumerate}
Przykładowa preambuła dokumentu wygląda następująco:
\begin{verbatim}
\documentclass{SGGW-thesis}
\title{Stworzenie klasy LaTeX-owej do użytku przy pisaniu
pracy dyplomowej w SGGW}
\author{Łukasz Adamczyk}
\date{2017}
\university{Szkoła Główna Gospodarstwa Wiejskiego\\w Warszawie}
\dep{Wydział Zastosowań Informatyki i Matematyki}
\Etitle{Creation of LaTeX class to use when writing a thesis at
the Warsaw University of Life Sciences -- SGGW}
\album{154833}
\thesis{Praca dyplomowa inżynierska}
\course{Informatyka}
\promotor{dr. hab. inż. Leszka Chmielewskiego, prof. SGGW}
\pworkplace{Wydział Zastosowań Informatyki i Matematyki\\
Katedra Informatyki\\Zakład Podstaw Informatyki}
\end{verbatim}
Należy zwrócić uwagę na to, że w argumentach makr mogą występować znaki nowej linii \verb|\\|, jeśli są potrzebne, na przykład gdy automatyczny podział długiego tytułu
pomiędzy linie tekstu nie jest korzystny. Znak nowej linii zawsze wystąpi w makrze \verb|\university|, gdzie zgodnie z~wytycznymi~\cite{wymagania} słowa {\em Szkoła Główna
Gospodarstwa Wiejskiego} są w~jednej linii, a~słowa {\em w Warszawie} w~następnej.

Często występująca konieczność wstawiania znaków nowej linii w tytułach prac jest przyczyną, dla której tytuły na stronie skrótów nie są pobierane z argumentów makr
generujących tytuły na stronie tytułowej, lecz trzeba je podać  osobno.


\section{Wnętrze środowiska document}
Środowisko \verb|document| zawiera treść pisanego dokumentu. Praca dyplomowa zaczyna się stroną tytułową, którą tworzymy komendą \verb|\maketitle|. Strona druga jest zależna
od ilości osób będących autorami pracy. W przypadku samodzielnego tworzenia strona ta pozostaje pusta. Dla pracy dwuosobowej należy użyć komendy \verb|\twoppage| przyjmującej
sześć parametrów:
\begin{enumerate}
\item{imię i nazwisko pierwszej osoby,}
\item{numer albumu pierwszej osoby,}
\item{numery rozdziałów pracy napisanych przez pierwszą osobę,}
\item{imię i nazwisko drugiej osoby,}
\item{numer albumu drugiej osoby,}
\item{numery rozdziałów pracy napisanych przez drugą osobę.}
\end{enumerate}
Gdy praca ma trzech autorów, należy skorzystać z komendy \verb|\threeppage| przyjmującej dziewięć parametrów. Pierwsze sześć z nich są identyczne jak w przypadku opisanej już
komendy \verb|\twoppage|, natomiast pozostałe to:
\begin{enumerate}\addtocounter{enumi}{6}
\item{imię i nazwisko trzeciej osoby,}
\item{numer albumu trzeciej osoby,}
\item{numery rozdziałów pracy napisanych przez trzecią osobę.}
\end{enumerate}
Trzecia strona pracy dyplomowej przeznaczona jest na oświadczenia. Stronę tą razem z następną pustą stroną generuję się komendą \verb|\statementpage|.

Na piątej stronie znajdują się streszczenia. Do jej stworzenia służy komenda o~nazwie \verb|\abstractpage| przyjmująca sześć parametrów:
\begin{enumerate}
\item{tytuł pracy w języku polskim,}
\item{treść streszczenia w języku polskim,}
\item{słowa kluczowe w języku polskim,}
\item{tytuł pracy w języku angielskim,}
\item{treść streszczenia w języku angielskim,}
\item{słowa kluczowe w języku angielskim.}
\end{enumerate}
Strona siódma zawiera spis treści. Wygenerować go można za pomocą standardowej komendy \LaTeX-a \verb|\tableofcontents|.

Przedostatnia strona pracy generowana jest komendą \verb|\beforelastpage|. W przypadku pracy przygotowanej lub współfinansowanej z udziałem interesariuszy zewnętrznych
postulujących klauzulę tajności należy posłużyć się wywołaniem z opcjonalnym argumentem zawierającym rok. Przykładowa komenda w takim przypadku to
\verb|\beforelastpage[2017]|. W~ten sposób do oświadczenia zostanie dopisane zastrzeżenie, że pracę można udostępnić po określonym roku.

Warto wspomnieć, że podczas pisania pracy do rozpoczynania rozdziałów należy użyć komendy \verb|\chapter|, a podrozdziałów \verb|\section|, ze względu na dziedziczenie tej
funkcjonalności z~klasy \verb|report|.


\chapter{Podsumowanie i wnioski}
Zgodnie z założeniami stworzono klasę ułatwiającą tworzenie dokumentów spełniających wymagania przedstawione w~\cite{wymagania}. Ostateczna implementacja zawiera niecałe
trzysta linii kodu źródłowego, co świadczy o zwięzłości klasy. Zastosowanie komentarzy znacznie poprawia czytelność. Z perspektywy piszącego, odpowiednio użyta klasa
zapewnia, że tekst spełnia wymagania uczelni stawiane pracy dyplomowej.


\begin{thebibliography}{9}
\bibitem{wymagania}
\textit{Zarządzenie nr 34 Rektora Szkoły Głównej Gospodarstwa Wiejskiego w Warszawie z dnia 01 czerwca 2016 r.\ w~ sprawie wprowadzenia ,,Wytycznych dotyczących
przygotowywania prac dyplomowych w~Szkole Głównej Gospodarstwa Wiejskiego w Warszawie''}, Załączniki 1 i~2
\url{https://www.sggw.pl/dla-studentow/informacje-formalno-prawne/dokumenty-do-pobrania}
$\rightarrow$ Praca dyplomowa (dostęp: 04.01.2017)
\bibitem{latexclass}
\textit{\LaTeXe\ for class and package writers} \url{http://www.latex-project.org/help/documentation/} $\rightarrow$ LaTeX2e for class... (dostęp 04.01.2017)
\bibitem{latexcompanion}
Frank Mittelbach and Michel Goossens with Johannes Braams, David Carlisle and Chris Rowley,
\textit{The \LaTeX\ Companion}. Second Edition,
Addison-Wesley, 2004.
\bibitem{tds}
\textit{A Directory Structure for \TeX\ Files} \url{http://tug.org/tds/tds.html} (dostęp 11.01.2017)
\end{thebibliography}

\beforelastpage

\end{document}
